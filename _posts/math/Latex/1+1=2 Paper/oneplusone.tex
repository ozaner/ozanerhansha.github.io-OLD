% A paper that seeks to construct, and thus prove, the notions of '1', '2', and '+'. (BUT IT COULD BE MORE! A PHILOSOPHICAL, PSYCHOLOGICAL, AND MATHEMATICAL UNDERTAKING OF THE VERY NOTION OF MEANING AND DEFINITOIN. TO KNOPW ANYTHING WE MUST HAVE BASE FACTS.)
\documentclass{article}
\usepackage{graphicx}

\begin{document}
\title{A Proof of 1+1=2}
\author{Ozaner Hansha}
\maketitle

\begin{abstract}
This paper aims to serve as a primer on the foundations and seemingly infinite regress of abstractness of which mathematics is based off. We do this by asking the following: ``Why does $1+1=2$?''

The answer is not immediately obvious and further in our investigation it becomes necessary to define what these symbols ($=,+,1,2$) even mean in the context of a greater framework of logic (in this case first order ZFC) to make sense of the statement.
\end{abstract}

\section{Introduction}
Much of mathematics is a matter of proof. But any proof, no matter how rigorous, can often be met with the same familiar question:

\begin{quotation}
  \begin{center}
    ``But why?''
  \end{center}
\end{quotation}

Indeed, any statement, mathematical or otherwise, can undergo a seemingly infinite regress from more abstract (high-level) statements to more concrete (low-level) statements.

A question a toddler might repeatedly ask his own parents

\begin{quotation}

\end{quotation}

But of course, this regress is very much finite, and at the end of this line of questions lie the goldpot. the big bang or ion the case of fol zfc an axiom

Take the following example:


But surely there must be an end? A

\begin{equation}
    \label{simple_equation}
    \alpha = \sqrt{ \beta }
\end{equation}

\subsection{Subsection Heading Here}
Write your subsection text here.

\section{Conclusion}
Write your conclusion here.

\end{document}
