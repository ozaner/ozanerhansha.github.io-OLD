\documentclass{article}
\usepackage{amssymb}
\usepackage{setspace}
\usepackage{csquotes}

\doublespacing
\begin{document}

\title{%
  On Bloom's ``Is God an Accident" \\
  \large Human Nature \& Diversity Paper}
\author{Ozaner Hansha}
\date{April 29, 2019}
\maketitle

\abstract{This paper has 4 main goals regarding the analysis of Paul Bloom's 2004 article ``Is God an Accident": 1) Explain the byproduct theory of religion as well as its older alternatives. 2) Elaborate on the research Bloom provides as justification for this theory. 3) Analyze and critique the theory and its associated evidence. 4) Address the theory's consequences, namely what it means for the (dis)belief of God.}

\section{The Byproduct Theory}
Bloom begins the article by noting that, regardless of country, creed, or culture, the majority of humans hold superstitious beliefs, particularly those pertaining to souls, gods, and the afterlife. Clearly then, this propensity to believe in religion is an aspect of human nature. The natural question that falls from this is why might this be the case. The `religion-as-an-accident' theory, or \textbf{byproduct theory}, seeks to provide an answer grounded in evolution and experimental evidence.

The core idea of the theory is that in reasoning about the objects of the world around them, humans split them into two broad categories: the \emph{physical} and the \emph{psychological}. Physical things, like rocks, trees, and airplanes, move and act in predictable ways according to known laws. Psychological things, like people and dogs, are unpredictable. They perform a whole host of actions that are hard to predict without relying on some sort of abstraction of their behaviors like emotions (e.g. currently angry, might attack) and goals (e.g. wanted ice cream, might go buy ice cream). This split in reasoning is neither cultural nor social, although those may reinforce it, but innate. And so explanations of complex or seemingly unanswerable questions are deferred to constructed psychological entities and processes.

What happens after death? We go to heaven or hell. What causes rain? The tears of the harvest god. And so on. The explanations, in particular supernatural ones, we come up with are byproducts of our dualist mind making sense of the world by attributing phenomena to some psychological entity, just like other complex/non-understood processes.

To both motivate and provide evidence for the theory, Bloom presents contemporary research (including some of his own) to us. But before we review this evidence, we will first briefly touch on two other theories that may also explain humankind's propensity for religion but, as we'll see, fail to explain its origins/supernatural flavor.

\subsection{Opiate Theory}
Before the byproduct theory came into prominence, other theories were proposed to explain the phenomena of religion amongst humans. A primary one being Marx's opiate theory. The idea is that religions form in order to 'sooth the pain of existence.' That is, dealing with our fear of death (e.g. afterlife), dealing with the inherent unfairness of life (e.g. karma), etc. In particular, Freud gave a religion in this framework a 3-fold task (Freud, 1927):

\begin{itemize}
	\item ``[T]hey must exorcise the terrors of nature,..."
	\item ``[T]hey must reconcile men to the cruelty of Fate, and..."
	\item ``[T]hey must compensate them for the sufferings and privations which a civilized life in common has imposed on them."
\end{itemize}

\subsection{Fraternity Theory}
Another alternative exists in the fraternity theory. It posits that the seemingly strange (e.g. circumcision) and even maladaptive (e.g. dietary restrictions) norms of religions are meant to create solidarity between members. This solidarity itself is an adaptive trait, allowing the group to become more cohesive, effective, and to support each other when needed. The theory also accounts for the harsh treatment of non-believers in many religions (e.g. excommunication, eternal damnation, etc.) as punishing those who don't participate in the group, and thus don't contribute to its success, also seems like an adaptive trait.

\subsection{Problems}
These theories, however, have some problems. Firstly, the opiate theory doesn't explain religions that do not fulfill, and may even exacerbate, those three criterion. There are plenty of societies that believe in spirits/gods that don't ensure fairness, the immortality of the soul, or anything of the sort. Moreover the theory doesn't explain the more mundane aspects of religion that seem to have no purpose in assuaging those three fears, e.g. halal/kosher foods, gender norms, etc.

The fraternity theory, while making sense of religious traditions as a means of creating solidarity, doesn't explain why these religious practices are religious. That is to say, it doesn't explain why the supernatural appears in all human cultures despite the purpose of religion being, supposedly, to build groups.

These problems are accounted for in the byproduct theory, whose evidence we will review in the next section. That said, the two other theories are still useful in explaining the allure and continued proliferation of religions, just not the origin of their supernatural aspects.

\section{Evidence of Innate Dualism}
If the byproduct theory is true, then we should see this dichotomous thinking manifest in all humans innately. An apt subject to test such a claim are babies. They are relatively untouched by culture and society, leaving only evolution to blame for their behaviors. Luckily for us, Bloom happens to be a developmental psychologist and has studied this very question (indeed his findings and others' compelled him to write this piece). Below we will address those contemporary psychologists' findings.

\subsection{In Infants}
While asking babies what they believe may be impossible, it is still possible to infer their thought processes via \textbf{looking-time experiments}. These are ubiquitous in the study of infant psychology. A looking-time experiment has a baby subject shown some control phenomena, e.g. an object falling after being dropped. Then, the baby is shown the experimental phenomena, e.g. an object not falling after being dropped (an illusion set up by the testers). How long the baby stares at the phenomena measures their surprise or unexpectedness of the event. In the given example, the babies would stare longer at the ball that didn't fall vs. the one that did since they have an innate sense of gravity.

Using a looking-time experiments, babies have been shown to understand object permanence and simple arithmetic. But, more crucially, they have been shown to be able to discern social cues like the emotions or the target of an adult's gaze. Notice that the former of the two infant abilities correspond to innate physical reasoning while the latter two correspond to innate psychological reasoning. Being able to assign mental states to other humans, even as an infant, is strong evidence that our inbuilt and dualist ascribing of psychological entities to external objects (like parents and puppies) is just that, inbuilt. A product of evolution and not learned.

Even further, evidence that this truly is a byproduct of a dualist mechanism, and not just a set of hardwired reactions to parents' facial expressions, can be found in studies that show babies an inanimate object `chasing' another. The results show that the baby ascribes intent to the chaser and chasee and expects them each to seek/avoid each other.

\subsection{In Young Children}
Another potent piece of evidence for the native dualist claim can be found in a study done on younger children. The subjects were told a story about an alligator that ate, and thus killed, a mouse. The subjects were then asked a variety of questions about what can and can't the mouse do/have happen to it. As expected, the children appreciated that biological (i.e. physical) properties no longer applied to the mouse (e.g. ``Does the mouse need to go to the bathroom? Does his ears work? His brain?" etc.). But when asked about mental (i.e. psychological) properties of the mouse, over half the subjects said that those properties still applied (e.g. ``Is the mouse hungry? Is he thinking about the alligator? Does he want to go home? etc.")

These children have an innate intuition that the `soul' survived the mouse's death, despite not knowing the specifics (if any) of what particular superstition their family/culture believes in. Whether that be a heaven/hell, reincarnation, or anything else. This tells us that their answers aren't a product of learned religious customs, but a byproduct of their dualist intuitions.

\section{Analysis}
Bloom's presentation of the byproduct theory as an alternative to previous and unsatisfactory, if not incomplete, theories on the origin of religion is very compelling. Seeing as the vast majority of humans believe in some religious superstition like God, the afterlife, souls, etc. (a fact Bloom made very clear at the beginning of the piece) it would make sense that if there was any explanation of religion's ubiquity in humans, that it would be based in our common evolutionary psychology.

The experimental results Bloom provides are also uncontroversial, at least nowadays, yet serve as clear evidence of our innate propensity to assign these so called \emph{psychological} properties to complex systems and problems.

\subsection{Intentional Systems}
In fact the dichotomy in our thinking, whose existence Bloom argues for, echoes the work of another influential philosopher of the mind: Daniel Dennett and his intentional systems theory (Dennett 1971). Indeed, humans think of so called \emph{psychological} objects in terms of their emotions, goals, temperament, etc. (i.e. in the \emph{intentional stance}) because doing so provides the most useful results. Trying to predict what a predator or fellow human might do next by considering the mechanics/physics of their brain and body is practically impossible and further, to a stone age human, unthinkable.

On the other hand, thinking of \emph{physical} objects in terms of its design, mechanics, external properties, etc. (i.e. in the \emph{physical/design stance}) is certainly more effective at predicating its behavior than as an intentional system. Consider a rock thrown in the air. Its behavior has nothing to do with any `goal' or `emotion' the rock may have been ascribed.

\subsection{An Adaptation}
So, viewed in this way, the byproduct theory seems like a very probable and useful adaptation. Because the more complex systems of humans and animals are harder to predict than ordinary physical things, it makes sense that we have evolved a heuristic way of dealing with them. That `way' being our notions of souls and personhoods, each of which has its own goals emotions and mental states. This is the basis of our own inbuilt `folk psychology' that provides  us a means to reason about others effectively in abstract ways rather than via clunky and unknown casual laws. As Bloom puts it, we are ``natural-born dualists."

\section{Evidence against God?}
The elephant in the room amidst all this talk on the origins of religion, at least to a believer (i.e. a product of this innate dualism), is the question of what this means for the belief in God.

I don't think I should waste time trying to argue against people's belief in the supernatural. After all, as the byproduct theory has shown us, belief in superstitions like God is a very deep rooted notion to us.

Indeed it seems almost inescapable, with a potent example being found in one of Bloom's contemporaries Justin Barret, who himself wrote an entire book titled ``Why Would Anyone Believe in God?" which uses much the same facts and findings Bloom does but in the \emph{opposite} direction, claiming that God gave us these propensities as a way to be able to find him (whatever finding him entails). Clearly, despite whatever evidence of the origin of man and his behaviors is given to us by anthropologists, evolutionary psychologists, and cognitive scientists, it pales in comparison to the combination of both our innate draw of dualist concepts combined with the indoctrination provided by our social and religious customs.

Put another way, the byproduct theory addresses a primary aspect of human nature and diversity: the ubiquity of religion and other superstitious beliefs. Despite this however, one's knowledge of this theory doesn't negate its affect on them. Consider a scientist who knows all there is to know about emotions and their properties: their causes, when they can be manipulated, etc. He may now be better equipped to reason about his own emotions, but he isn't necessarily immune to its pitfalls. In the same way, teaching humans about their own nature won't necessarily make them change their behavior or admit their own inconsistencies.

Even in the face of the myriad of inconstancies of any particular religion, (e.g. the three O's of the Abrahamic God) and its clear evolutionary and cultural origins (any major religious text is clearly a product of its time, reflecting its cultural norms and notions of morality) the byproduct theory would suggest that, bar some large social upheaval, people will continue to believe in superstition. Performing the mental gymnastics necessary to feel justified in doing so, or even skipping that and simply denying any evidence to the contrary. It's just human nature, and there's no reason to expect it to change anytime soon.

\begin{thebibliography}{999}
\bibitem{dm}
  Bloom, Paul\\
  ``Is God an Accident?", \emph{The Atlantic}, 2005
\bibitem{dm}
  Dennett, Daniel\\
  \emph{Intentional Systems}, \emph{The Journal of Philosophy}, 1971
\bibitem{dm}
  Marx, Karl\\
  \emph{Critique of Hegel's Philosophy of Right}, 1843
\bibitem{dm}
  Freud, Sigmund\\
  \emph{The Future of an Illusion}, 1972
\bibitem{dm}
  Bering, Jesse; Bjorklund, David\\
  ``The Natural Emergence of Reasoning About the Afterlife as a
  Developmental Regularity", \emph{Developmental Psychology Vol. 40}, 1940
\end{thebibliography}
\end{document}
