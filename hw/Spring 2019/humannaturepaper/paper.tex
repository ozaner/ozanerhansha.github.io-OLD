\documentclass{article}
\usepackage{amssymb}
\usepackage{setspace}
\usepackage{csquotes}

\doublespacing
\begin{document}

\title{%
  On Bloom's ``Is God an Accident" \\
  \large Human Nature \& Diversity Paper}
\author{Ozaner Hansha}
\date{April 29, 2019}
\maketitle

\abstract{This paper has 4 main goals regarding the analysis of Paul Bloom's 2004 article ``Is God an Accident": 1) Explain Bloom's ``religion-as-an-accident" theory as well as its immediate alternatives. 2) Elaborate on the evidence Bloom provides as justification for this theory. 3) Analyze and critique the theory and its associated evidence. 4) Address the theory's consequences, namely what it means for the (dis)belief of God.}

\section{Religion as an Accident}
Bloom beings the article by noting that, regardless of country, creed, or culture, the majority of humans hold superstitious beliefs, in particular those pertaining to souls, gods, and the afterlife. Clearly, then, this propensity to believe in religion must be an aspect of human nature. This motivates his so called ``religion-as-an-accident" theory which seeks to provide an evolutionary explanation of its origin.

The core idea of the theory is that in reasoning about the objects of the world around them, humans split them into two broad categories: the \emph{physical} and the \emph{psychological}. Physical things, like rocks, trees, and airplanes, move and act in predictable ways according to known laws. Psychological things, like people and dogs, are unpredictable. They have goals, emotions, and perform a whole host of other actions that are hard to predict without relying on some sort of theory of mind.

To this end, Bloom provides contemporary research (some of which is even his own) as evidence to the theory. But before we review this evidence, we will first briefly touch on some other theories that may also explain humankind's propensity for religion but, as we'll see, fail to explain its origins/supernatural flavor.

\subsection{Similar Theories}

\section{The Evidence}


\section{Analysis}

\subsection{Intentional Systems}
This dichotomy is very similar, possibly identical, to that of functional and intentional systems in Dennett's (Dennett 1971) intentional systems theory. Indeed, humans think of so called \emph{psychological} objects in terms of their emotions, goals, temperament, etc. (i.e. in the \emph{intentional stance}) because doing so provides the most useful results. Trying to predict what a predator or fellow human might do next by considering the mechanics/physics of their brain and body is practically impossible and further, to a stone age human, unthinkable.

On the other hand, thinking of \emph{physical} objects in terms of its design, mechanics, external properties, etc. (i.e. in the \emph{physical/design stance}) is certainly more effective at predicating its behavior than as an intentional system. Consider a rock thrown in the air. Its behavior has nothing to do with any `goal' or `emotion' the rock may have been ascribed.

\section{Evidence against God?}
this propensity seems inescapable, with a glaring example being found in one of Bloom's contemporaries Justin Barret, who himself wrote an entire book titled "why would anyone beleive in god" which uses much the same facts and findings Bloom does, but in the *opposite* direction, claiming that God gave us these propensities as a way to be able to find him (whatever that entials). claerly despite whatever evidence of the origin of man and his behviors given to us by anthropolosigtgs, evolutionary psychologists, and cognitive scientists, it pales in comparison to the power of the intuitive feelings we have of 'souls' 'gods' and all things mystical and unfounded.


\begin{thebibliography}{999}
\bibitem{dm}
  Bloom, Paul\\
  ``Is God an Accident?", \emph{The Atlantic}, 2005
\bibitem{dm}
  Dennett, Daniel\\
  \emph{Intentional Systems}, \emph{The Journal of Philosophy}, 1971
\end{thebibliography}
\end{document}
