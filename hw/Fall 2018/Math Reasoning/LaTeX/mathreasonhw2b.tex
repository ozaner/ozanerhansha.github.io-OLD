\documentclass{article}
\usepackage{amsmath}
\usepackage{tikz}

\begin{document}

\title{Intro to Math Reasoning HW 2b}
\author{Ozaner Hansha}
\date{September 20, 2018}
% RUID: 188003484
\maketitle

\section{Problem 1}
\textbf{Problem:} Give a positive integer $k$ that satisfies $2^k> k^{1000} + 10^6$
\\\\
\textbf{Solution:} Looking at the inequality we can already tell that there must be some $k$ for which it is true. This is because by looking at limiting behavior of $2^k$ and $k^{1000}+10^6$, we see that the former grows much faster than the latter (and that the constant $10^6$, although large, disappears as $k\to\infty$).

So then where does the left hand side of the inequality take over the right hand side? Well let us pick some large value, say $k=10^7$. Plugging this in we see:

\begin{align*}
2^{10^7}&>(10^7)^{1000}+10^6\\
&>10^{7000}+10^6
\end{align*}

Notice that due to the magnitudes we are dealing with (which will be illustrated below), the constant $10^6$ is irrelevant and so we will omit it:

$$2^{10000000}>10^{7000}$$

Written out, it seems more clear that the left hand side is clearly bigger. However we can go further and show it must be true by taking the $\log_2$ of both sides (a valid operation as the $\log$ respects the real's total ordering):

\begin{align*}
\log_22^{10000000}&>\log_210^{7000}\\
10000000&>7000\log_210\\
10000000&>7000\cdot(3.322)\\
10000000&>23253.5
\end{align*}

\textit{Note that some rounding occurred in calculating $\log_210$}\\

With the last statement certainly being true.

\section{Problem 2}
\textbf{Problem:} Give an infinite sequence $\left(A_j\right)_{j=1}^\infty$ that satisfies the following properties for all $j\in\mathbb{Z}_{>0}:$
\begin{gather*}
A_j\not=\emptyset\\
A_j\subseteq [-100,100]\\
 A_{j+1}\subseteq A_j\\
\bigcap_{j=1}^{\infty}A_j=\emptyset
\end{gather*}
\\
\textbf{Solution:} To solve this we will construct a sequence of subsets of the interval $[-100,100]$ with sizes that decrease at each step but never reach 0 for any finite step. To that end, note the following equality:
\begin{gather*}
1+\frac{1}{2}+\frac{1}{4}+\frac{1}{8}+\cdots+\frac{1}{2^i}+\cdots=2\\
\downarrow\text{Multiply convergent series by -100}\\
-100-50-25-\cdots=-200\\
\downarrow\text{Add 200 to both sides}\\
200-100-50-25-\cdots=0
\end{gather*}

The above equation reminds us that an infinite number of terms can add up to 0. In the same sense, we will construct the following sets:
\begin{align*}
A_1&=[0,200]\\
A_2&=[0,100]\\
A_3&=[0,50]\\
&\ \ \vdots\\
A_k&=[0,\frac{200}{2^k}]\\
&\ \ \vdots
\end{align*}
Notice that each set is a subset of the previous, and that for any element in the reals $\epsilon$ there exists some $A_\delta$ such that $\epsilon\not\in A_\delta$.

Now we can just adapt this for our interval which is shifted by -100:
\begin{align*}
A_1&=[-100,100]\\
A_2&=[-100,0]\\
A_3&=[-100,-50]\\
&\ \ \vdots\\
A_k&=[-100,-100+\frac{100}{2^{k-2}}]\\
&\ \ \vdots
\end{align*}

\section{Problem 3}
\subsection{Part a}
\textbf{Problem:} Give a partition of $\mathbb{Z}_{>0}$ with 2 parts of infinite size.
\\\\
\textbf{Solution:} The partition with one part even and one part odd. We can state this as $\{E,O\}$ where:
\begin{align*}
E&=\{x\in\mathbb{Z}_{>0}\mid (2x)\}\\
O&=\{x\in\mathbb{Z}_{>0}\mid (2x+1)\}
\end{align*}

\subsection{Part b}
\textbf{Problem:} Show how one could construct a partition of $\mathbb{Z}_{>0}$ into $k\in\mathbb{Z}_{>0}$ parts, all of infinite size.
\\\\
\textbf{Solution:} In part a we knew that every positive integer either had a factor of 2 or not. This time we will sort numbers into parts based off of the prime numbers after 2.

Consider the set of all prime numbers up to $k-1$. We will call this (indexed) set $(p_j)_{j=1}^{k-1}$ where $p_j$ is the $j$th prime number. Now let us define a collection of sets $(S_j)_{j=1}^{k-1}$:
$$S_j=\{p_j^x\mid x\in\mathbb{Z}_{>0}\}$$
\\
\indent That is, the $S_j$ is the set of all powers of the $j$th prime number. All of which are distinct from the powers of other prime numbers. Now all that's left is to define the final set $S_k$ to include all the other positive integers that are not powers of primes (of which there are infinitely many):

$$S_k=(\cdots(\mathbb{Z}_{>0}\setminus S_1)\setminus S_2)\setminus\cdots)\setminus S_{k-1}$$

Or in other words, all elements of the positive integers that are not in $S_1$ to $S_{k-1}$.

And we are done. The sets from $S_1$ to $S_k$ form a partition of the positive integers with an infinite number of parts each with an infinite number of elements.

\subsection{Part c}
\textbf{Problem:} Partition the positive integers into an infinite amount of infinite sized parts.
\\\\
\textbf{Solution:} This is just a generalization of the above. Consider the infinite (indexed) set of all prime numbers $(p_j)_{j=1}^{\infty}$. Now construct a family of sets $S_j$ that uses the same definition as in part b. Now simply construct another set $S_{\omega}$ that is the positive integers minus every set in the family $S_j$.

\textit{Explicit set subtractions won't work as there are an infinite amount of sets. We will need an arbitrary set subtraction.}

The set of all $S_j$ and $S_\omega$ would form a partition that satisfies the above condition.

\section{Problem 4}
\subsection{Part a}
\textbf{Problem:} Give a permutation of the set $\{1,2,3,4,5\}$
\\\\
\textbf{Solution:} Since the elements of the set can also be thought of the indices of a list, I will just write out a list that can be interpreted as a permutation: $(1,3,4,2,5)$.

\subsection{Part b}
\textbf{Problem:} Give 2 distinct permutations $f,g$ of the above set such that neither is the identity permutation yet $f\circ g=g\circ f$.
\\\\
\textbf{Solution:} Notice that this means $f$ and $g$ are inverses of each other. Again I will use the list notation:
\begin{align*}
f&=(3,1,2,5,4)\\
g&=(2,3,1,5,4)
\end{align*}

\subsection{Part c}
\textbf{Problem:} Give 2 distinct permutations $f,g$ of the above set such that neither is the identity permutation and $f\circ g=g\circ f$ and $f\circ f=g\circ g$.
\\\\
\textbf{Solution:} Here are two permutations that satisfy the above:
\begin{align*}
f&=(3,1,4,2,5)\\
g&=(2,4,1,3,5)
\end{align*}

Solving out the compositions we find that $f\circ f=g\circ g=(4,3,2,1,5)$.

\section{Problem 5}
Let S denote the set $\{2, 10, 11, 18, 19, 27\}$.
\subsection{Part a}
\textbf{Problem:} For all elements $x$ belonging to $S$ there is a $y$ belonging to $S$ such that $x + y$ is a multiple of 5.
\\\\
\textbf{Solution:} This is true, and we can prove it via exhaustion:
\begin{gather*}
5\mid (2+18)\\
5\mid (10+10)\\
5\mid (11+19)\\
5\mid (27+18)
\end{gather*}
Notice that, because addition is commutative, it suffices for every element in $S$ to appear in parenthesis once overall the equations.

\subsection{Part b}
\textbf{Problem:} There exists an element $x$ belonging to $S$ such that for all elements $y$ belonging to $S$, $x + y$ is a multiple of 5.
\\\\
\textbf{Solution:} This statement is false. There is no single element in $S$ that, when added by any element in $S$, is always divisible by 5. One way to prove this is to simply add 2 to every element:

$$\{4,10,13,20,21,29\}$$

and also add 11 to every element:

$$\{13,21,22,29,30,38\}$$

Notice that while there is an element in both sets that is a multiple of 5, the 20 in the first set arose from the element 18 while the 30 in the bottom arose from the element 19. Thus there is no single element in $S$ that when added by any particular element in $S$ produces a multiple of 5.

\subsection{Part c}
\textbf{Problem:} For all $x$ belonging to $S$ there is a $y$ belonging to $S$ such that $x + y$ is a multiple of 7.
\\\\
\textbf{Solution:} This is true. Every element can be added with some other element to form a number divisible by 7:
\begin{gather*}
7\mid (2+19)\\
7\mid (10+11)\\
7\mid (27+18)
\end{gather*}
Again, addition is commutative so these 3 examples suffice for all 6 elements.

\end{document}
