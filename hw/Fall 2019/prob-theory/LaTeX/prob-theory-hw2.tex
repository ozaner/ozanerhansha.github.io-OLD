\documentclass{article}
\usepackage{amsmath}
\usepackage{amssymb}
\usepackage[dvipsnames]{xcolor}
\usepackage[margin=1.2in]{geometry}

\begin{document}

\title{Theory of Probability HW \#2}
\author{Ozaner Hansha}
\date{September 23, 2019}
\maketitle

\begin{center}
    \Large{\textbf{Part A}}
\end{center}
Problems taken from Chapters 2 and 3 of the textbook.

\section*{\underline{Chapter 2}}

\section*{Problem 31a}
\noindent\textbf{Problem:} A 3-person basketball team consists of a guard, a forward, and a center. If a person is chosen at random from each of three different such teams, what is the probability of selecting a complete team?
\bigskip

\noindent\textbf{Solution:} The number of valid choices from the first team is 3 (since we have none yet), 2 for the second team (1 position is taken), and 1 from the third team (2 positions are taken). Putting this over the $3^3$ different choices we have:

\begin{equation*}
    P(E)=\frac{3\cdot2\cdot 1}{3^3}=\frac{2}{9}
\end{equation*}

\section*{Problem 45}
\noindent\textbf{Problem:} 
\bigskip

\noindent\textbf{Solution:} 
\bigskip

\section*{Problem 55a}
\noindent\textbf{Problem:} 
\bigskip

\noindent\textbf{Solution:}

\section*{\underline{Chapter 3}}

\section*{Problem 12}
\noindent\textbf{Problem:} 
\bigskip

\noindent\textbf{Solution:} 
\bigskip

\section*{Problem 30}
\noindent\textbf{Problem:} 
\bigskip

\noindent\textbf{Solution:} 
\bigskip

\section*{Problem 35}
\noindent\textbf{Problem:} 
\bigskip

\noindent\textbf{Solution:} 
\bigskip

\section*{Problem 47a}
\noindent\textbf{Problem:} 
\bigskip

\noindent\textbf{Solution:} 

\begin{center}
    \Large\textbf{Part B}
\end{center}

\section*{Problem a}
noindent\textbf{Problem:} 
\bigskip

\noindent\textbf{Solution:}
\bigskip

\section*{Problem b}
noindent\textbf{Problem:} 
\bigskip

\noindent\textbf{Solution:}
\bigskip

\section*{Problem c}
noindent\textbf{Problem:} 
\bigskip

\noindent\textbf{Solution:} 

\end{document}