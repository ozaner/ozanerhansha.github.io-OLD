\documentclass{article}
\usepackage{amssymb}
\usepackage{setspace}
\usepackage{csquotes}

\doublespacing
\begin{document}

\title{%
  On Clark's ``Without Evidence or Argument" \\
  \large Intro to Philosophy Paper \# 1}
\author{Ozaner Hansha}
\date{October 1, 2018}
\maketitle

\abstract{There are 3 main goals of this paper. 1) Extract the argument Kelly J. Clark posits in his article ``Without Evidence or Argument" that Clifford's argument is false. 2) Elaborate on Clark's justification of said argument. 3) Analyze and critique this argument and its associated justification.}

\section{Extraction}
We will extract the argument from the paragraph starting with ``The first problem..." in conjunction with the paragraph after it. Before we do this however we should note that in the paragraphs preceding the argument, Clark begins to question the enlightenment principles of reason and evidence as bases for belief. In particular, Clark characterizes Clifford's argument for the morality of evidence based beliefs as reflective a child pestering his parents and even infantile (indirectly via William James). Clearly, Clark feels strongly about his argument. Especially in its relation to this commonly held principle. The crux of this strong belief is that Clifford's principle, he claims, cannot itself be proved from reason and evidence. His argument can be formulated in the following way:

\begin{itemize}
    \item[1.] All rational beliefs must be proved from evidence.
    \item[2.] What we may take as evidence is our sensory experience and self-evident truths.
    \item[3.] There is no way to prove Premise 1 from our sensory experiences and self-evident truths.
    \item[$\therefore$] Therefore, by Premise 1, belief in Premise 1 is irrational.
\end{itemize}

\section{Justification}
\subsection{Premise 1}
The reasoning for this is quite straightforward. Premise 1 is what Clark interprets Clifford's argument's conclusion to be. We won't get into Clifford's argument but Clark clearly doesn't believe it holds for \textit{all} rational beliefs. Indeed, he explicitly states his disbelief in the generality of this premise:
\begin{displayquote}\textit{``No one would disagree: some beliefs require evidence for their rational acceptability. But \textnormal{all} beliefs in \textnormal{every} circumstance? That’s an exceedingly strong claim to make and, it turns out, one that cannot be based on evidence.''}
\end{displayquote}

Clark uses Premise 1, despite his disbelief in it, so that he can show that it is inconsistent with itself.

\subsection{Premise 2}
Again, Clark is taking to be true Clifford's own ideas of evidence and truth (at least in Clark's own view) to, hopefully, show that they are internally inconsistent. Here he states what many would find to be a reasonable depiction of the `evidence' Clifford refers to in his own argument: self-evident truths and sensory experience. Clark states logic and math are two examples of `self-evident truths'. In terms of sensory experience, we can assume that this covers 1) things not provable from math/logic like chemistry, biology or psychology and other things known from experiment and 2) our own everyday, unscientific observations like that ``the sky is blue, grass is green, most trees are taller than most grasshopper'', etc. to give Clark's own examples.

\subsection{Premise 3}
Clark doesn't give much justification for this premise other than that
\begin{displayquote}\textit{``None of the propositions that are allowed as evidence have anything at all to do with the conclusion.''}
\end{displayquote}

What he means, I assume, is that the premises we take to be true because they fall into the category of evidence ($2+2=4$, the brain is comprised of two hemispheres, etc.) have no clear connection to the idea that a rational belief in something requires evidence. Clark then takes this further to say that there exists \textit{no} connection between the set of all evidence and Clifford's principle.


\section{Analysis}
Before I point out why I disagree with Premise 1, I will state why I agree with Premises 2 and 3:
\subsection{Agreement with Premise 2}
As far the term `evidence' goes, I would say this is a good common definition between both arguments. It jives well with both Clark's and Clifford's argument, with the former asserting that it is not necessary to hold rational beliefs and the latter stating that it is. And most importantly it, at least would appear, stays true to the usage of the word in Clifford's argument. This is key to Clark's goal of disproving his argument.

\subsection{Agreement with Premise 3}
While I do agree with the justification of this premise, my agreement with it highlights what I believe to be a conceptual error on Clark's part which I will elaborate more on in my objection to Premise 1. But essentially, I agree that Premise 1 cannot be proven from evidence. This is because it is a matter of principle and depends on what we may consider rational (this of course bypasses many epistemological nuances which I have no experience with nor time to detail).

\subsection{Objection to Premise 1}
Clark asserts that Premise 1, that all rational beliefs must be proved from evidence, is the conclusion Clifford came to in his `of-cited' paper. While I agree I would say that Clark is overextending the use of `rational' to have it include any belief that a human might hold. That is to say when Clifford says that God cannot be rationally believed in, because there is no evidence for it, that does not disprove God nor give reason for a human to not believe in God. It merely states that no human can have a belief in God based on truth (the kind we might normally consider like science and math).

While I agree that Clifford's premise that all beliefs are important might be a bit of an overstatement and that refuting it would disprove his conclusion, i.e Premise 1, this wouldn't help Clark. Clark is establishing this argument to show that God needs no evidence to be rationally believed. But God is certainly an important belief (and I suspect Clark would agree with this), and so this refutation doesn't salvage anything.

Another point is that Clifford's argument is foremost about the \textif{ethicality} of beliefs. If a belief is not rational (that is it is not based on evidence) than making decision that would affect the wellbeing of other based on this belief is morally wrong. If I were to have a belief not based on evidence yet still call it rational, this would mean I am justified in believing in it. If I was justified in this belief than I could certainly make decision regarding other's wellbeing. Say I believed in Christian interpretation of God, then it would behoove me to try and convert as many people as possible as it would benefit both me and them (good deeds). Notice that another, equally rational person, could believe in another belief not founded on evidence (like a satanic worshipper who also wants converts) and have conflicting ideas with the first person despite them both having, supposedly, rational justified beliefs.


\section{Conclusion}
In conclusion, I find Clark's argument lacking in that it doesn't use the same notion of rational, meaning justified, as Clifford's does. Clifford's example of the ships sailing were intended to evoke a sense of ethical responsibility, that sending people off into potential danger without evidence was wrong. Clark certainly does not disagree with this, but at the same time states that rational belief do not have to be grounded on evidence. This leads to situations in which we certainly CAN send people to their doom simply because we are making decisions not based on evidence.
\end{document}

\textit{From here on out I will refer to the statement ``All beliefs must be based on evidence" as \textbf{Clifford's principle}.}

clifford offers us a principle, namely that evidence is required for rational b eleif in any propositoin. he did not prove this principle from inherent truths or espitemical evidence. nor does he expect anyone else too. instead this is offered as an explination of our own behaviors blah blah idk... what is WRONg to belive is an opinion, so then what is the relationship between opinion and truth (both inherent and epistemcial)? cliffords principle.

2 responsese
1 opinion guided principle
2. cant prove clifford principle but cant prove god either. why not have principle be god. equally as arbitrary.

clifford getting at why we believe vs ewhat we should belive RATIONALLY

conslcusion
unseuccesful attempt at atacking cliffords aarguemnt due to confusion between opinion and truth as well as not making a distinciont between rational belifs and irrational belids. also human behaviors vs princple

your own argument doesnt suffice (mirroring his objectio to clifforf) because it wasnt nescarry to establish your point appaerntly
