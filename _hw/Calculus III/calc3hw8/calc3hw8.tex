\documentclass{article}
\usepackage{amsmath}
\usepackage{amssymb}

\begin{document}

\title{Honors Calculus III HW \#8}
\author{Ozaner Hansha}
\date{November 26, 2018}
\maketitle

\section*{Exercise 1}
\textbf{Problem:} Consider the following function:
$$f(x,y)=xy+2x-2y$$
Find the maximum and minimum values of $f$ in the region that lies below $y=1-x^2$ and above $y=0$.
\\\\
\textbf{Solution:} The gradient is $\nabla f(x,y)=(y+2,x-2)$, meaning the only critical point is $(2,-2)$. Plugging this point in tells us it is outside the region, so we now only need to check the region's boundary.

The first part of this boundary is given by $b_1(x,y)=x^2+y-1$ when $b_1(x,y)=0$. By finding the gradient $\nabla b_1(x,y)=(2x,-1)$ we can give the following Lagrange equations:
\begin{gather*}
  y+2=2x(x-2)\\
  y=1-x^2
\end{gather*}

By plugging the second equation into the first we are left with $3x^3-4x-3=0$. This quadratic polynomial has the following solutions:
$$r=\frac{2\pm\sqrt{13}}{3}$$

Only the $-$ case of the root lied on the boundary so we need not consider the other. Plugging this root back into either of our original equations we can solve for $y$ as well, giving us the complete critical point:
$$\mathbf x_1=\left(\frac{2-\sqrt{13}}{3},\frac{4\sqrt{13}-8}{9}\right)$$

The second part of the boundary is given by $b_2(x,y)=y$ when $b_2(x,y)=0$. The gradient is then $\nabla b_2(x,y)=(0,1)$. Clearly, $\nabla f=\lambda\nabla b_2$ means that $y$ must equal $-2$. This isn't possible however as it falls outside of the bounded region. And so we need not consider this case any further.

Now we have the third case where the two previous boundaries intersect. These are the points $\mathbf x_2=(-1,0)$ and $\mathbf x_3=(1,0)$. Now we can see which ones are the max/mins:
\begin{align*}
  f(\mathbf x_1)&=\frac{16-26\sqrt{13}}{27}\approx-2.88\\
  f(\mathbf x_2)&=-2\\
  f(\mathbf x_3)&=2
\end{align*}

And so $\mathbf x_1$ is the minimizer which gives $\frac{16-26\sqrt{13}}{27}$. The maximizer is $\mathbf x_3$ which gives $2$

\section*{Exercise 2}
\textbf{Problem:} The intersection of the cone $z^2=x^2+y^2$ and the following plane is an ellipse:
$$2z=x+9$$
Find the points on the ellipse closest and farthest to the origin.
\\\\
\textbf{Solution:} This is equivalent to maximizing/minimizing the square distance $f(x,y,z)=x^2+y^2+z^2$ on the following boundaries when they equal 0:
\begin{align*}
  b_1(x,y,z)&=x-2z+9\\
  b_2(x,y,z)&=z^2-x^2-y^2
\end{align*}

Given that the ellipse is closed and bounded, there must be minimizers and maximizers. Lagrange's theorem tells us that these max/mins $\mathbf x$ must satisfy:
$$\nabla f(\mathbf x)\cdot\nabla b_1(\mathbf x)\times\nabla b_2(\mathbf x)=0$$

And so, made explicit, this gives us:
$$(2x,2y,2z)\cdot(1,0,-2)\times(-2x,-2y,2z)=-8yz=0$$

This means that $y=0$ or $z=0$. If $y=0$ then the equations $g_1=0$ and $g_2=0$ become:
\begin{align*}
  x-2z+9&=0\\
  z^2-x^2&=0
\end{align*}

We first solve for $x$ in the first equation and plug it into the second giving us $z^2-12z+27=0$ which has roots $z=3,9$. Plugging these back in we find their corresponding $x$ values to be $-3$ and $9$ respectively. This gives us the following critical points:
\begin{align*}
  \mathbf x_1=(-3,0,3)\\
  \mathbf x_2=(9,0,9)
\end{align*}

If $z=0$ then the equations $g_1=0$ and $g_2=0$ become:
\begin{align*}
  x+9&=0\\
  -x^2-y^2&=0
\end{align*}

This system, however, has no real solutions. And so we are left to plug our critical points into the function to find:
\begin{align*}
  f(\mathbf x_1)&=3\sqrt 2\\
  f(\mathbf x_2)&=9\sqrt 2
\end{align*}

And so $\mathbf x_1$ is the minimizer (closest to origin) and $\mathbf x_2$ is the maximizer (farthest from origin).

\section*{Exercise 3}
Consider the following function:
$$f(x,y,z)=x+yz$$
And now consider the following two equations which together give a curve in $\mathbb R^3$:
\begin{gather*}
  3x^2+4y^2+z^2=1\\
  z=x^2+y^2
\end{gather*}
\subsection*{Part a}
\textbf{Problem:} Give the system of 3 equations that all maximizers/minimizers of $f$ must satisfy due to Lagrange's Theorem:
\\\\
\textbf{Solution:} Consider the following functions:
\begin{align*}
  b_1(x,y,z)&=3x^2+4y^2+z^2-1\\
  b_2(x,y,z)&=z-x^2-y^2
\end{align*}

The gradients of all these functions come out to:
\begin{align*}
  \nabla f(x,y,z)&=(1,z,y)\\
  \nabla b_1(x,y,z)&=(6x,8y,2z)\\
  \nabla b_2(x,y,z)&=(-2x,-2y,1)
\end{align*}

Lagrange's theorem tells us that the triple product of these gradients must be 0:
$$4x(y^2-z^2)+z(4y-6x)+8y=0$$

This equation, along with the two original constraints comprise the system of 3 equations we originally sought after.

\subsection*{Part b}
\textbf{Problem:} One of these 3 equations can be used to reduce this system of equations to 2 variables. In particular, eliminate $z$ from this system.
\\\\
\textbf{Solution:} If we substitute the second constraint equation (i.e. $z=x^2+y^2$) into the first two, we arrive at the following system of 2 equations:
\begin{align*}
  4x^3(x^2-2y^2)+4xy^4+6x^3-4y^3+2xy^2-4x^2y-8y&=0\\
  3x^2+4y^2+(x^2+y^2)^2-1&=0
\end{align*}

\end{document}
