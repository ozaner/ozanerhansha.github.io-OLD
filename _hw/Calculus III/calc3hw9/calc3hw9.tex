\documentclass{article}
\usepackage{amsmath}
\usepackage{amssymb}

\begin{document}

\title{Honors Calculus III HW \#9}
\author{Ozaner Hansha}
\date{December 5, 2018}
\maketitle

\section*{Exercise 1}
\textbf{Problem:} Using the mean value theorem, prove Cauchy's Mean value theorem. That is, for two functions $f,g$ continuous on the interval $[a,b]$ and differentiable on $(a,b)$, there exists a $c$ with $a<c<b$ such that:
$$g'(c)(f(b)-f(a))=f'(c)(g(b)-g(a))$$
\textbf{Solution:} First let us consider the following function:
$$h(x)=f(x)(g(b)-g(a))-g(x)(f(b)-f(a))$$

Note that this function is also continuous and differentiable on the same intervals as $f$ and $g$. Also note that the mean value theorem guarantees that for some $c$ that satsifies $a<c<b$:
$$h'(c)=\frac{h(b)-h(a)}{b-a} $$

However, notice that when evaluating $h(b)-h(a)$ we arrive at:
\begin{align*}
  h(b)-h(a)&=f(b)g(b)-f(b)g(a)-g(b)f(b)+g(b)f(a)\\
  &-f(a)g(b)-f(a)g(a)-g(a)f(b)+g(a)f(a)\\
  &=0
\end{align*}

And since the numerator is zero the whole of $h'(c)=0$. Also notice that when we differentiate $h(x)$ and plug in $c$ we get the following:
$$h'(c)=f'(c)(g(b)-g(a))-g'(c)(f(b)-f(a))$$

But because $h'(c)=0$ we can say:
$$f'(c)(g(b)-g(a))=g'(c)(f(b)-f(a))$$

or if the denominators aren't zero:
$$\frac{f(b)-f(a)}{g(b)-g(a)}=\frac{f'(c)}{g'(c)}$$

And so the $c$ we knew existed via the MVT satisfies the more general Cauchy MVT.

\section*{Exercise 2}
\textbf{Problem:} Consider a continuous function $f$ on an open interval containing $[a,b]$ where $f'$ and $f''$ exist on $(a,b)$. Show that if $x\in(a,x)$ then there exists a $c\in(a,x)$ such that:
$$f(x)=f(a)+f'(a)(x-a)+\frac{1}{2}f''(c)(x-a)^2$$
\textbf{Solution:} First consider the following continuous and differentiable functions, where $a,b$ and $f$ are the same as above:
\begin{align*}
  h(x)&=f(x)-f(a)-f'(a)(x-a)\\
  g(x)&=(x-a)^2
\end{align*}

Cauchy's MVT guarantees that there exists some $d$ such that:
$$\frac{h(x)-h(a)}{g(x)-g(a)}=\frac{h'(d)}{g'(d)}$$

Note that $d$ satisfies $a<d<x$. Now we if we apply Cauchy's MVT again, but this time on the functions $h'(x)$ and $g'(x)$ over the interval $[a,d]$, we get:
$$\frac{h'(x)-h'(a)}{g'(x)-g'(a)}=\frac{h''(c)}{g''(c)}$$

Since $h'(a)=g'(a)=0$ transitivity gives us:
$$\frac{h(x)-h(a)}{g(x)-g(a)}=\frac{h''(c)}{g''(c)}$$

Also note that $a$ is clearly a root of both $h$ and $g$ leaving us with:
$$\frac{h(x)}{g(x)}=\frac{h''(c)}{g''(c)}$$

First we will list the second derivatives of $h$ and $g$:
\begin{align*}
  h'(x)=f'(x)-f'(a)\\
  h''(x)=f''(x)\\
  g'(x)=2(x-a)\\
  g''(x)=2
\end{align*}

Plugging these in we find:
\begin{align*}
  \frac{f(x)-f(a)-f'(a)(x-a)}{(x-a)^2}&=\frac{f''(c)}{2}\\
  f(x)-f'(a)(x-a)&=\frac{f''(c)}{2}(x-a)^2\\
  f(x)&=f'(a)(x-a)+\frac{f''(c)}{2}(x-a)^2
\end{align*}

And we are done.

\section*{Exercise 3}
\subsection*{Part a}
\textbf{Problem:} Suppose a function $f$ has continous second partials. Using Taylor's Theorem, show that there exists an $s\in(0,t)$ that satisfies the following equation on the interval $[0,t]$ for some choice of $\mathbf x_0$ and $\mathbf v$:
$$f(\mathbf x_0+t\mathbf v)=f(\mathbf x_0)+t(\nabla f(\mathbf x_0)\cdot\mathbf v)+\frac{t^2}{2}Hf(\mathbf x_0+s\mathbf v)\mathbf v\cdot\mathbf v$$
\textbf{Solution:} Taylor's Theorem tells us that for some $s\in(0,a)$:
$$g_\mathbf v(t)=g_\mathbf v(0)+g'_\mathbf v(0)(t-0)+\frac{1}{2}g''_\mathbf v(s)$$

Now recall that the first derivative of a function from $\mathbb R^n\to\mathbb R^m$ parameterized by a line (i.e. let $t$ vary in $\mathbf x_0+t\mathbf v$) is given by it's Jacobian matrix applied to $\mathbf v$. Also note that the Jacobian reduces to the gradient when $m=1$ (i.e. an $n\times 1$ Jacobian):
$$g'_\mathbf v(t)=Jf(\mathbf x_0+t\mathbf v)\mathbf v=\nabla f(\mathbf x_0+t\mathbf v)\cdot\mathbf v$$

Also note that the second derivative of a function from $\mathbb R^n\to\mathbb R$ is given by the Hessian matrix (i.e. the Jacobian of the Jacobian). So, just applying the same process as above twice over we get:
$$g''_\mathbf v(t)=J(Jf(\mathbf x_0+t\mathbf v)\mathbf v)\mathbf v=J(\nabla f(\mathbf x_0+t\mathbf v)\cdot\mathbf v)\mathbf v=Hf(\mathbf x_0+t\mathbf v)\mathbf v\cdot\mathbf v$$

And so now if we simply plug $f$ back in for $g$, and do a little bit of simplifying, we indeed find:
$$f(\mathbf x_0+t\mathbf v)=f(\mathbf x_0)+t(\nabla f(\mathbf x_0)\cdot\mathbf v)+\frac{t^2}{2}Hf(\mathbf x_0+s\mathbf v)\mathbf v\cdot\mathbf v$$

\subsection*{Part b \& c}
\textbf{Problem:} Rewrite the above expression with the following substitutions:
\begin{align*}
  \mathbf x_0+t\mathbf v&=\mathbf x=\mathbf x_0+|\mathbf x-\mathbf x_0|\left(\frac{\mathbf x-\mathbf x_0}{|\mathbf x-\mathbf x_0|}\right)\\
  s&=r|\mathbf x-\mathbf x_0|
\end{align*}
Then manipulate the resulting expression to the following:
\begin{align*}f(\mathbf x)-f(\mathbf x_0)-(\nabla f(\mathbf x_0)\cdot\mathbf x-\mathbf x_0)-\frac{1}{2}Hf(\mathbf x_0)(\mathbf x-\mathbf x_0)\cdot(\mathbf x-\mathbf x_0)\\
=\frac{1}{2}(Hf(\mathbf x_0+r(\mathbf x-\mathbf x_0))-Hf(\mathbf x_0))(\mathbf x-\mathbf x_0)\cdot(\mathbf x-\mathbf x_0))
\end{align*}
\textbf{Solution:} Plugging in these values we see:
\begin{align*}
  f(\mathbf x)&=f(\mathbf x_0)+|\mathbf x-\mathbf x_0|(\nabla f(\mathbf x_0)\cdot \frac{\mathbf x-\mathbf x_0}{|\mathbf x-\mathbf x_0|})+\frac{t^2}{2}Hf(\mathbf x_0+s\mathbf v)\mathbf v\cdot\mathbf v\\
  &=f(\mathbf x_0)+(\nabla f(\mathbf x_0)\cdot\mathbf x-\mathbf x_0)+\frac{|\mathbf x-\mathbf x_0|^2}{2}Hf\left(\mathbf x_0+r|\mathbf x-\mathbf x_0|\left(\frac{\mathbf x-\mathbf x_0}{|\mathbf x-\mathbf x_0|}\right)\right)\left(\frac{\mathbf x-\mathbf x_0}{|\mathbf x-\mathbf x_0|}\right)\cdot\left(\frac{\mathbf x-\mathbf x_0}{|\mathbf x-\mathbf x_0|}\right)\\
  &=f(\mathbf x_0)+(\nabla f(\mathbf x_0)\cdot\mathbf x-\mathbf x_0)+\frac{1}{2}Hf(\mathbf x_0+r(\mathbf x-\mathbf x_0))(\mathbf x-\mathbf x_0)\cdot(\mathbf x-\mathbf x_0)\\
\end{align*}

Now let us manipulate just the last term:
\begin{align*}
&\ \ \ \ \frac{1}{2}Hf(\mathbf x_0+r(\mathbf x-\mathbf x_0))(\mathbf x-\mathbf x_0)\cdot(\mathbf x-\mathbf x_0)\\
&=\frac{1}{2}(Hf(\mathbf x_0)+Hf(r(\mathbf x-\mathbf x_0))(\mathbf x-\mathbf x_0))\cdot(\mathbf x-\mathbf x_0)\\
&=\frac{1}{2}(Hf(\mathbf x_0)(\mathbf x-\mathbf x_0)+Hf(r(\mathbf x-\mathbf x_0))(\mathbf x-\mathbf x_0)))\cdot(\mathbf x-\mathbf x_0)\\
&=\frac{1}{2}Hf(\mathbf x_0)(\mathbf x-\mathbf x_0)\cdot(\mathbf x-\mathbf x_0)+\frac{1}{2}Hf(r(\mathbf x-\mathbf x_0))(\mathbf x-\mathbf x_0)\cdot(\mathbf x-\mathbf x_0))\\
&=\frac{1}{2}Hf(\mathbf x_0)(\mathbf x-\mathbf x_0)\cdot(\mathbf x-\mathbf x_0)+\frac{1}{2}(Hf(r(\mathbf x-\mathbf x_0))+Hf(\mathbf x_0)-Hf(\mathbf x_0))(\mathbf x-\mathbf x_0)\cdot(\mathbf x-\mathbf x_0))\\
&=\frac{1}{2}Hf(\mathbf x_0)(\mathbf x-\mathbf x_0)\cdot(\mathbf x-\mathbf x_0)+\frac{1}{2}(Hf(\mathbf x_0+r(\mathbf x-\mathbf x_0))-Hf(\mathbf x_0))(\mathbf x-\mathbf x_0)\cdot(\mathbf x-\mathbf x_0))\\
\end{align*}

Now if we just plug this back in and subtract the first 3 terms to the left hand side we are left with:
\begin{align*}f(\mathbf x)-f(\mathbf x_0)-(\nabla f(\mathbf x_0)\cdot\mathbf x-\mathbf x_0)-\frac{1}{2}Hf(\mathbf x_0)(\mathbf x-\mathbf x_0)\cdot(\mathbf x-\mathbf x_0)\\
=\frac{1}{2}(Hf(\mathbf x_0+r(\mathbf x-\mathbf x_0))-Hf(\mathbf x_0))(\mathbf x-\mathbf x_0)\cdot(\mathbf x-\mathbf x_0))
\end{align*}

\subsection*{Part d}
\textbf{Problem:} Use the above result and the Cauchy-Schwartz Inequality to show that:
$$\frac{f(\mathbf x)-f(\mathbf x_0)-(\nabla f(\mathbf x_0)\cdot\mathbf x-\mathbf x_0)-\frac{1}{2}Hf(\mathbf x_0)(\mathbf x-\mathbf x_0)\cdot(\mathbf x-\mathbf x_0)}{|\mathbf x-\mathbf x_0|^2}\le \frac{1}{2}|Hf(\mathbf x_0+r(\mathbf x-\mathbf x_0))-Hf(\mathbf x_0)|$$
\\\\
\textbf{Solution:} The Cauchy-Schwartz Inequality tells us that $|A\mathbf v|\le|A||\mathbf v|$. So if we just apply this twice, once for the matrix multiply by $(\mathbf x-\mathbf x_0)$ and once for the dot product by $(\mathbf x-\mathbf x_0)$, which is just matrix multiplication transposed, we can divide the left hand side by this norm twice and are left with the desired statement.

\subsection*{Part e}
\textbf{Problem:} Prove continuous second partials imply second order differentiability:
$$\lim_{\mathbf x\to\mathbf x_0}\frac{f(\mathbf x)-f(\mathbf x_0)-(\nabla f(\mathbf x_0)\cdot\mathbf x-\mathbf x_0)-\frac{1}{2}Hf(\mathbf x_0)(\mathbf x-\mathbf x_0)\cdot(\mathbf x-\mathbf x_0)}{|\mathbf x-\mathbf x_0|^2}=0$$
\textbf{Solution:} Recall that we have already shown:
$$0\le\cdots\le\frac{1}{2}|Hf(\mathbf x_0+r(\mathbf x-\mathbf x_0))-Hf(\mathbf x_0)|$$

Also note that the limit of the right-hand side as $\mathbf x\to\mathbf x_0$ is 0:
$$\lim_{\mathbf x\to\mathbf x_0}\frac{1}{2}|Hf(\mathbf x_0+r(\mathbf x-\mathbf x_0))-Hf(\mathbf x_0)|=\frac{1}{2}|Hf(\mathbf x_0)-Hf(\mathbf x_0)|=0$$

And so by the squeeze theorem the desired statement is true. (this automatically implies the statement on the sheet, we just didn't need an explicit construction of $\delta$)

\end{document}
