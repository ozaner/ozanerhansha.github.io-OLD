\documentclass{article}
\usepackage{amsmath}
\usepackage{amssymb}
\usepackage{listings}

\begin{document}

\title{Numerical Analysis HW \#2}
\author{Ozaner Hansha}
\date{March 7, 2019}
\maketitle

\section*{Problem 1}
\subsection*{Part a}
\textbf{Problem:}
\textbf{Solution:}

\begin{center}
\begin{tabular}{r|c|c}
      Iteration & $x_n$ & $y_n$\\
      \hline
      0 & 0.500000000000000 & 0.500000000000000\\
      1 & 1.000000000000000 & 0.500000000000000\\
      2 & 0.812500000000000 & 0.437500000000000\\
      3 & 0.773719879518072 & 0.420557228915663\\
      4 & 0.771848952636680 & 0.419645658001209\\
      5 & 0.771844506371371 & 0.419643377620421\\
      6 & 0.771844506346038 & 0.419643377607081\\
      7 & 0.771844506346038 & 0.419643377607081
\end{tabular}
\end{center}


\section*{Problem 5}
\textbf{Problem:} Solve the same system as in problem 4 but use Matlab's \verb|fsolve| routine to do it. In particular, use the following code:

\begin{lstlisting}[language=Matlab]
options = optimset('Display', 'iter');
x0 = [0.5,0.5]
[x,fval] = fsolve(@fcnns,x0,options)|
\end{lstlisting}
\textbf{Solution:}
Running this code, with the system defined in \verb|fccns.m|, returns the following approximation and error:

\begin{lstlisting}[language=Matlab]
x = 0.771844506371479   0.419643377620486

fval = 1.0e-10 *

  -0.250830467507512
  -0.146170298087611
\end{lstlisting}

Note that \verb|fsolve| defaults to 10 decimal places of accuracy ($\operatorname{tol}=10^{-10}$) and so it only agrees with problem 4's answer to the 10th decimal place.

\end{document}
