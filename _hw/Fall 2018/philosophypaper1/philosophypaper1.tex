\documentclass{article}
\usepackage{amssymb}
\usepackage{setspace}
\usepackage{csquotes}

\doublespacing
\begin{document}

\title{%
  On Clark's ``Without Evidence or Argument" \\
  \large Intro to Philosophy Paper \# 1}
\author{Ozaner Hansha}
\date{October 1, 2018}
\maketitle

\abstract{There are 3 main goals of this paper. 1) Extract the argument Kelly J. Clark posits in his article ``Without Evidence or Argument" that Clifford's argument is false. 2) Elaborate on Clark's justification of said argument. 3) Analyze and critique this argument and its associated justification.}

\section{Extraction}
\subsection{Some Context}
We will extract the argument from the paragraph starting with ``The first problem..." in conjunction with the paragraph after it. Before we do this however we should note that in the paragraphs preceding the argument, Clark begins to question the enlightenment principles of reason and evidence as bases for belief. In particular, Clark characterizes Clifford's argument for the morality of evidence based beliefs as reflective of a child pestering his parents and even infantile (indirectly via William James). Clearly, Clark feels strongly about his argument. Especially in its relation to this commonly held principle. The crux of this strong belief is that Clifford's principle, he claims, cannot itself be proved from reason and evidence.

\subsection{The Argument}
The argument can be formulated in the following way:

\begin{itemize}
    \item[1.] Clifford's view holds that all rational beliefs must be proved from evidence.
    \item[2.] What we may take as evidence, in Clifford's view, is our sensory experience and self-evident truths.
    \item[3.] There is no way to prove Premise 1 from our sensory experiences and self-evident truths.
    \item[$\therefore$] Therefore, by Premise 1, belief in Premise 1 is an irrational belief in Clifford's view.
\end{itemize}

\section{Justification}
\subsection{Premise 1}
The reasoning for this is quite straightforward. Premise 1 is what Clark interprets as the conclusion of Clifford's argument. We won't get into Clifford's argument but Clark clearly doesn't believe it holds for \textit{all} rational beliefs. Indeed, he explicitly states his disbelief in the generality of this premise:
\begin{displayquote}\textit{``No one would disagree: some beliefs require evidence for their rational acceptability. But \textnormal{all} beliefs in \textnormal{every} circumstance? That’s an exceedingly strong claim to make and, it turns out, one that cannot be based on evidence.''}
\end{displayquote}

Clark uses Premise 1, despite his disagreement with it, so that he can show that it is inconsistent with itself.

\subsection{Premise 2}
Again, Clark is taking to be true Clifford's own ideas of evidence and truth (at least in Clark's own view) to, hopefully, show that they are internally inconsistent. Here he states what many would find to be a reasonable depiction of the `evidence' Clifford refers to in his own argument: self-evident truths and sensory experience. Clark states logic and math are two examples of `self-evident truths'. In terms of sensory experience, we can assume that this covers 1) things not provable from math/logic like chemistry, biology or psychology and other things known from experiment and 2) our own everyday, unscientific observations like that ``the sky is blue, grass is green, most trees are taller than most grasshopper'', etc. to give Clark's own examples.

\subsection{Premise 3}
Clark doesn't give much justification for this premise other than that
\begin{displayquote}\textit{``None of the propositions that are allowed as evidence have anything at all to do with the conclusion.''}
\end{displayquote}

What he means, I assume, is that the premises we take to be true because they fall into the category of evidence ($2+2=4$, the brain is comprised of two hemispheres, etc.) have no clear connection to the idea that a rational belief in something requires evidence. Clark then takes this further to say that there exists \textit{no} connection between the set of all evidence and Clifford's principle.


\section{Analysis}
Before I point out why I disagree with Premise 1 and 3, I will state why I agree with Premise 2:
\subsection{Agreement with Premise 2}
As far the term `evidence' goes, I would say this is a good common definition between both arguments. It jives well with both Clark's and Clifford's argument, with the former asserting that it is not necessary to hold rational beliefs and the latter stating that it is. And most importantly it, appears to at least, stay true to the usage of the word in Clifford's argument. This is key to Clark's goal of disproving his argument.

\subsection{Disagreement with Premise 1}
My objection to Premise 1, and the entire argument in general, is that Clark is not using the same definition of `rational belief' as Clifford does. It would seem that Clark took the word to mean \textit{any} belief someone may reasonably hold rather than Clifford's more limited definition of any belief someone may ethically hold. Clifford's argument is first and foremost about the ethicality of belief. He holds that it is morally wrong for people to hold beliefs that would affect others without evidence.

For example, I can believe that I am going to win the lottery if I wanted, but making decisions based on this belief (like preemptively purchasing a new house) that may affect others is certainly unethical. This is because I have no reason (i.e. evidence) to believe I will win, in fact I have much evidence to the contrary. I doubt Clark would disagree with this, and so it's likely he was using a different notion of `rational'.

One may object with the fact that Clifford takes this statement even farther and states that \textit{every} belief is important enough to affect others. This would imply that every belief must be founded on evidence to be ethically held. I would, and I imagine Clark would as well, say that this is most likely too far of a generalization. But regardless of whether this part of Clifford's argument is correct or not, it is irrelevant to Clark's position as the belief in Premise 1 is certainly important enough to affect others and so too is the belief in God (with the defense of God's existence being the ultimate goal of Clark).

Here's an example to illustrate the contradiction of rational beliefs if they did not need to be founded on evidence: Say I believed in Christian interpretation of God. It would behoove me to try and convert as many people as possible as it would benefit both me and them (good deeds). My belief in this fact is rational (at least in Clark's mind). Now consider a Satanic worshipper whose belief in the worship of Satan, not being founded on evidence, is just as rational as any other belief. It too would behoove them to convert as many people as possible. The goals of these two groups are entirely contradictory and this means that at least one of them is trying to do something that is negatively affecting others (either sending people to eternal damnation via the will of God/Satan or just wasting their time).

Clark could still salvage God as a rational belief (in Clifford's view) if he can prove God's existence from evidence. However I doubt he believes that this is possible.

\subsection{Disagreement with Premise 3}
My disagreement with Premise 3 is just a consequence of, what is in my view, Clark's generalization of the word rational. Indeed, I think it would be very possible to 'prove' Premise 3 simply via the copious amount of observations people can make regarding the success of decisions based on rational versus irrational beliefs.

That said, my objection may be a bit too strong as some might say that there is a difference between truly \textit{proving} something and simply showing that it is very likely to be true (via mounds of evidence). And while I might agree with that, the small chance that our many observations, and even confidence in self-evident truths, may actually be erroneous would suggest that such a strong definition of \textit{prove} is not so useful in this discussion.

\section{Conclusion}
Summarizing, my main objection to Clark's argument is that it doesn't use the same notion of rational, meaning ethically justified, as Clifford's does. Clifford's example of the ships sailing were intended to evoke a sense of ethical responsibility, that sending people off into potential danger without evidence was wrong. Clark probably does not disagree with this, but at the same time states that rational belief do not have to be grounded on evidence. This leads to situations in which we certainly \textit{can} send people to their doom simply because we are making decisions not based on evidence. Either Clark believes this, or he was simply using a different definition of rational than Clifford. I'm assuming the latter rather than the former.

\begin{thebibliography}{999}
\bibitem{clark}
  Kelly J. Clark,
  ``Without Evidence or Argument: A Defense of Reformed Epistemology''.

\bibitem{clifford}
  William K. Clifford,
  ``The Ethics of Belief'', \emph{Contemporary Review}, 1877.
\end{thebibliography}
\end{document}

\textit{From here on out I will refer to the statement ``All beliefs must be based on evidence" as \textbf{Clifford's principle}.}
