\documentclass{article}
\usepackage{amsmath}
\usepackage{amssymb}

\begin{document}

\title{Intro to Math Reasoning HW 4a}
\author{Ozaner Hansha}
\date{October 3, 2018}
\maketitle

\section*{Problem 1}
\textbf{Problem:} Consider the predicate $C(x,y)$ where $x$ and $y$ are real numbers. Let the sets $S_1$ and $S_2$:
\begin{align*}
S_1&=\{x\in\mathbb R\mid(\forall y\in\mathbb R)\ C(x,y)\}\\
S_2&=\{y\in\mathbb R\mid(\forall x\in\mathbb R)\ \neg C(x,y)\}
\end{align*}
Can both $S_1$ and $S_2$ be nonempty?
\\\\
\textbf{Solution:} No, either $S_1$ or $S_2$ must be empty. We can make this clearer by renaming $x$ and $y$ in the definition of $S_2$:

\begin{align*}
S_1&=\{x\in\mathbb R\mid(\forall y\in\mathbb R)\ C(x,y)\}\\
S_2&=\{x\in\mathbb R\mid(\forall y\in\mathbb R)\ \neg C(y,x)\}
\end{align*}

They both cannot be nonempty if $C(x,y)\equiv C(y,x)$. But even if this didn't hold, they still coudln't be nonempty. This is because whatever the predicate is, it has to hold for all $y$ in $S_1$ and not hold for all $y$ in $S_2$. This means we would have to be able to distinguish between $x$ and $y$ in the predicate but if we could then whatever held for all in one case wouldn't in the other.

% Yes, and to prove this it suffices to give a single example. Consider the following choice of $C(x,y)$:
%
% $$C(x,y)\equiv y\not=0\wedge x=0$$
%
% We can see that $S_1$ is the set $\{x\mid\}$

% Yes, and to prove this it suffices to give a single example. Consider the following choice of $C(x,y)$:
%
% $$C(x,y)\equiv (\exists! k\in\mathbb R)\ kx=y$$
%
% Notice that this is just a more accurate way of saying $\frac{y}{x}$ is uniquely defined. We know that for all real numbers $x,y$ the statement is defined and thus the proposition is true \textit{except} when $x=0$ since $\frac{y}{0}$ is not defined for any $y$.
% \\\\
% So in this case $S_1=\{x\in\mathbb R\mid x\not=0\}$ and $S_2=\{0\}$ which are both nonempty.

\section*{Problem 2}
Consider the predicate $P(A,B,C)\equiv (C\setminus A=C\setminus B)\rightarrow A=B$.
\subsection*{Part a}
\textbf{Problem:} Is there an $A,B$ and $C$ such that $P(A,B,C)$ is true?
\\\\
\textbf{Solution:} Yes there is. $A=\{1\}$, $B=\{1\}$, and $C=\{1,2,3\}$
\begin{gather*}
  C\setminus A=\{2,3\}=C\setminus B\\
  A=\{1\}=B
\end{gather*}

Both the antecedent and the consequent are true, thus the predicate is satisfied.

\subsection*{Part b}
\textbf{Problem:} Is there a unique $(A,B,C)$ such that $P(A,B,C)$ is true?
\\\\
\textbf{Solution:} No. It suffices to show two examples of this. One was shown above, another is $A=\{1\}$, $B=\{2\}$, and $C=\{1,2,3\}$
\begin{gather*}
  C\setminus A=\{1\}\not=C\setminus B=\{2\}\\
  A=\{1\}\not=B=\{2\}
\end{gather*}

The antecedent is false and the consequent is false, thus the predicate is satisfied.

\subsection*{Part c}
\textbf{Problem:} Is there an $A,B$ and $C$ such that $P(A,B,C)$ is false?
\\\\
\textbf{Solution:} Yes there is. $A=\{0,1\}$, $B=\{1\}$, and $C=\{2,3\}$
\begin{gather*}
  C\setminus A=\{2,3\}=C\setminus B=\{2,3\}\\
  A=\{0,1\}\not=B=\{1\}
\end{gather*}

The antecedent is true and the consequent is false, thus the predicate is not satisfied.

\section*{Problem 3}
\textbf{Problem:} For all sets $A,B$, and $C$ is the following true:
$$A\cup B\subseteq C\implies (C\setminus(A\cup B) = (C\setminus A)\cap (C\setminus B))$$
\textbf{Solution:} The proposition is true. To prove it let us first note the definition of the antecedent:
\begin{align*}
  A\cup B &= \{x\mid x\in A \vee x\in B\}\\
  A\cup B\subseteq C&\equiv (x\in A\cup B)\implies x\in C\\
  &\equiv (x\in A \vee x\in B)\implies x\in C
\end{align*}

We can rename the atomic propositions in the above compound proposition like so:
$$(a \vee b)\rightarrow c$$

Now we do the same for the first half of the consequent:
\begin{align*}
  A\cup B &= \{x\mid x\in A \vee x\in B\}\\
  C\setminus(A\cup B)&= \{x\mid x\in C\wedge \neg(x\in A\cup B)\}\\
  &= \{x\mid x\in C\wedge \neg(x\in A \vee x\in B)\}
\end{align*}

Like above, we can rename the propositions like so:
$$c\wedge\neg(a \vee b)$$

Finally we do the same for the second half of the antecedent:
\begin{align*}
  C\setminus A &= \{x\mid x\in C \wedge x\not\in A\}\\
  C\setminus B &= \{x\mid x\in C \wedge x\not\in B\}\\
  (C\setminus A)\cap (C\setminus B)&= \{x\mid (x\in C\setminus A) \wedge (x\in C\setminus B)\}\\
  &= \{x\mid (x\in C \wedge x\not\in A) \wedge (x\mid x\in C \wedge x\not\in B)\}
\end{align*}

Again, we can rename the propositions like so:
\begin{align*}
  (c\wedge\neg a)\wedge(c\wedge\neg b)&\equiv c\wedge(\neg a\wedge\neg b) \tag{distributive property}\\
  &\equiv c\vee\neg (a\vee b)\tag{De Morgan's law}
\end{align*}

Now we simply have to prove the following:

$$((a \vee b)\rightarrow c)\rightarrow (c\wedge\neg(a \vee b)\equiv c\wedge\neg (a\vee b))$$

However notice that the consequent is trivially a tautology (it is literally the same expression on both sides). Since the consequent is always true, the truth of the antecedent is irrelevant and the statement as a whole is true.

\section*{Problem 4}
\textbf{Problem:} For all sets $A,B$, and $C$ is the following always true:
$$A\cup B\subseteq C\implies C\setminus(A\cup B)\subseteq C\setminus A$$
\textbf{Solution:} Yes this is true and we can prove it in the same way as above. Start with the antcedent:
\begin{align*}
  A\cup B &= \{x\mid x\in A \vee x\in B\}\\
  A\cup B\subseteq C&\equiv (x\in A\cup B)\implies x\in C\\
  &\equiv (x\in A \vee x\in B)\implies x\in C\\
  &\equiv a\vee b\rightarrow c
\end{align*}

Now the consequent
\begin{align*}
  C\setminus(A\cup B)&= \{x\mid x\in C\wedge \neg(x\in A\cup B)\}\\
  &\equiv \{x\mid x\in C\wedge \neg(x\in A \vee x\in B)\}\\
  &\equiv c\wedge \neg(a\vee b)\\
  C\setminus A&=\{x\mid x\in C\wedge x\not\in A\}\\
  &\equiv c \wedge \neg a\\
  C\setminus(A\cup B)\subseteq C\setminus A &= (x\in C\setminus(A\cup B))\rightarrow (x\in C\setminus A)\\
  &\equiv c\wedge \neg(a\vee b)\rightarrow c\wedge\neg a
\end{align*}

So now we just have to prove the following:
$$(a\vee b\rightarrow c)\rightarrow (c\wedge \neg(a\vee b)\rightarrow c\wedge\neg a)$$

However consider the consequent, which is itself an implication:
\begin{align*}
  &\phantom{\equiv\ \ } c\wedge \neg(a\vee b)\rightarrow c\wedge\neg a\\
  &\equiv c\wedge (\neg a\wedge \neg b) \rightarrow c\wedge\neg a\tag{De Morgan's law}\\
  &\equiv (c\wedge \neg a)\wedge \neg b \rightarrow c\wedge\neg a\tag{associative property}\\
\end{align*}

With the last statement clearly being a tautology (simplification of a conjunction). Remember that the tautology above is the consequent of the bigger statement that we set out to prove. As a result of this, the statement we set out to prove is also a tautology, since its consequent is always true.

\section*{Problem 5}
\subsection*{Part a}
\textbf{Problem:} Is the following true proposition always true:

$$(A\rightarrow B)\rightarrow C \equiv A\rightarrow (B\rightarrow C)$$
\\\\
\textbf{Solution:} No. Here's a truth table:
\begin{center}
\begin{tabular}{ccccccc}
$A$ & $B$ & $C$ & $A\rightarrow B$ & $B\rightarrow C$ & $(A\rightarrow B)\rightarrow C$ & $A\rightarrow (B\rightarrow C)$\\
\midrule
\hline
F&F&F&T&T&F&T\\
F&F&T&T&T&T&T\\
F&T&F&T&F&F&T\\
F&T&T&T&T&T&T\\
T&F&F&F&T&T&F\\
T&F&T&F&T&T&T\\
T&T&F&T&F&F&F\\
T&T&T&T&T&T&T\\
\end{tabular}
\end{center}
As we can see the left and right hand propositions are not equivalent thus the proposition is false.
\subsection*{Part b}
\textbf{Problem:} Is it the case that either $(A\rightarrow B)\rightarrow C$ or $A\rightarrow (B\rightarrow C)$ must be true?
\\\\
\textbf{Solution:} No. Here's a truth table:
\begin{center}
\begin{tabular}{cccccc}
$A$ & $B$ & $C$ & $(A\rightarrow B)\rightarrow C$ & $A\rightarrow (B\rightarrow C)$ & $(A\rightarrow B)\rightarrow C \oplus A\rightarrow (B\rightarrow C)$\\
\midrule
\hline
F&F&F&F&T&T\\
F&F&T&T&T&F\\
F&T&F&F&T&T\\
F&T&T&T&T&F\\
T&F&F&T&F&T\\
T&F&T&T&T&F\\
T&T&F&F&F&F\\
T&T&T&T&T&F\\
\end{tabular}
\end{center}
The exclusive disjunction of the two statements indeed does not form a tautology. And so the statement we set out to disprove is false.
\end{document}
